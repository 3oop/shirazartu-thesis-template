% the introduction section...

\chapter{ارزیابی} \label{chapter:evaluation}

\paragraph*{}
داده‌های استفاده شده در این بخش توسط پژوهش میدانی انجام شده در تابستان 1401 بدست آمده‌است..

\section{کیفیت نمونه‌ها}

با توجه به تنوع جهان شمول نمونه‌های موزه، حفظ کیفیت تمام بخش‌ها به علت دسترسی نداشتن به نمونه‌های جدید سخت است. اما حتی اکثر نمونه‌های تاکسیدرمی شده از اکوسیستم بومی  متعلق به دهه 50 و 60 شمسی می‌باشند. قعالیت کارگاه تاکسیدرمی به مربوط به گونه‌های کمی ‌می‌شود و شیوه مدیریت این مسئله مانعی برای جذابیت های بالقوه موزه است.

\section{کیفیت ارائه}

نکات فنی در طراحی دیوراماها به خوبی رعایت شده بود اما تصاویر دکور با کیفیت نقاشی دکور ها اکثرا پشم نوار نبود.
نمونه‌های گیاهی در تمام دیوراماها بلااستثنا خشک شده‌بودند.
نمونه های فسیلی فضای کافی نداشتند.
تابلوی پوست آخرین ببر ایرانی، مانند اضافه جهیزیه در خانه‌ای در مرحله خانه‌تکانی گوشه‌ای افتاده بود.