% the proposed method section...

\chapter{روش پیشنهادی} \label{chapter:proposed-method}

\paragraph*{}
برای بررسی عملکرد موزه نیاز به مجموعه معیار هایی برای مقایسه داریم.


\section{بررسی موزه}
برای بررسی عملکرد موزه معیارهایی انتخاب می‌کنیم و در فصل بعد به مقایسه آن ها می‌پردازیم.

\subsection{ساختمان موزه}
اکثر موزه‌های داخل ایران درون ساختمان هایی که کاربری پیشینی متفاوتی داشتند برپا شدند. از جمله عمارت ها و کاخ های قدیمی.
بنابراین ویژگی‌های ساختمانی خاص مورد نیاز یک موزه را دارا نیستند.


\subsection{کامل بودن مجموعه}
موزه های تاریخ طبیعی شامل نمونه‌های جانوری، حشرات، آبزیان، گیاهان، فسیل‌ها و سنگ‌ها و کانی‌ها می‌شوند.


\subsection{نمونه‌ها}
یکی از عوامل منتخب برای ارزیابی، کیفیت نمونه‌هاست. بالا بودن کیفیت نمونه‌ها با شباهت به نمونه زنده مشخص می‌شود. این مسئله در کنار افزایش دقت اطلاعاتی که ببینده ارائه می‌کند، باعث جذابیت بیشتر بازدید از موزه می‌شود. 


\subsection{دیوراماها}
نمونه‌های تاکسیدرمی شده، در دیوراماها به نمایش گذاشته می‌شوند. ممکن است به چند گونه که زیست بوم مشترکی دارند یک دیوراما اختصاص داده شود.

دیوراماها با تصاویری از زیست بوم گونه آراسته می‌شوند. تصاویر باید ویژگی های معرف آن زیست بوم را داشته باشند و در معرفی اقلیم دقت داشته باشند. ممکن است از پوشش گیاهی آن زیست بوم نیز استفاده شود.

\subsection{نورپردازی}



\section{بررسی فعالیت‌ها}

\subsection{فعالیت‌های آموزشی}
نسبت به جمعیت سایر بازدید کنندگان، دانش آموزان از مخاطبان اصلی موزه هستند. معمولا بواسطه اردوهای آموزشی مدارس بازدید گروهی برایشان ترتیب داده می‌شود.
با توجه به مجهز بودن موزه به دپارتمان آموزش، این امر از اهداف اصلی موزه می‌باشد. 




