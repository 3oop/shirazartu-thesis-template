% the proposed method section...

\chapter{روش پیشنهادی} \label{chapter:proposed-method}

\paragraph*{}
برای بررسی عملکرد موزه نیاز به مجموعه معیار هایی برای مقایسه داریم.

\section{معیارها}
برای بررسی عملکرد موزه معیارهایی انتخاب می‌کنیم و در فصل بعد به مقایسه آن ها می‌پردازیم.

\subsection{کامل بودن مجموعه}
موزه های تاریخ طبیعی شامل نمونه‌های جانوری، حشرات، آبزیان، گیاهان، فسیل‌ها و سنگ‌ها و کانی‌ها می‌شوند.


\subsection{کیفیت نمونه‌ها}
یکی از عوامل منتخب برای ارزیابی، کیفیت نمونه‌هاست. بالا بودن کیفیت نمونه‌ها با شباهت به نمونه زنده مشخص می‌شود. این مسئله در کنار افزایش دقت اطلاعاتی که ببینده ارائه می‌کند، باعث جذابیت بیشتر بازدید از موزه می‌شود. 


\subsection{کیفیت ارائه}

نمونه‌های تاکسیدرمی شده، در دیوراماها به نمایش گذاشته می‌شوند.
ممکن است به چند گونه که زیست بوم مشترکی دارند یک دیوراما اختصاص داده شود.

دیوراماها با تصاویری از زیست بوم گونه آراسته می‌شوند. تصاویر باید ویژگی های معرف آن زیست بوم را داشته باشند و در معرفی اقلیم دقت داشته باشند. ممکن است از پوشش گیاهی آن زیست بوم نیز استفاده شود.

\subsection{معیارهای آموزشی}
نسبت به جمعیت سایر بازدید کنندگان، دانش آموزان از مخاطبان اصلی موزه هستند. معمولا بواسطه اردوهای آموزشی مدارس بازدید گروهی برایشان ترتیب داده می‌شود.
با توجه به مجهز بودن موزه به دپارتمان آموزش، این امر از اهداف اصلی موزه می‌باشد. 



