% the introduction section...

\chapter{سابقه‌ی تحقیق} \label{chapter:related-works}

\paragraph*{}
در مقدمه‌ی این فصل ابتدا انواع سامانه‌های ذخیره‌ای اطلاعات موجود و مورد استفاده در دنیا معرفی می‌گردند و هر یک با توجه به خصوصیاتی که دارا هستند، با یکدیگر مورد مقایسه قرار می‌گیرند. در ادامه موردهای استفاده‌ی هر یک از این انواع سامانه‌ها معرفی می‌شوند و با نگاه به تفاوت‌ها و خصوصیات هر یک، کاربردی که به نسبت دیگر کاربردها بیشتر مورد استفاده دارد انتخاب می‌شود. پس از این انتخاب، سامانه‌ی ذخیره‌ای داده‌ی متناسب با این کاربرد برای اعمال بهبود معرفی می‌شود.

\paragraph*{}
پس از مشخص شدن سامانه‌‌ی ذخیره‌ی داده و کاربرد منتخب برای بهبود، تاریخچه‌ی تحقیق در زمینه‌ی مشخص شده یک به یک بیان می‌شود. این بیان شامل مدل ارائه شده، معماری پیشنهادی مورد در صورت ارائه‌ی معماری، روش بهبود صورت گرفته و همچنین نواقص و ایرادات است. هدف از این فصل ارائه‌ی سابقه‌ی تحقیق از ابتدا و سیر تکامل روش‌های ذخیره‌ی داده در پایگاه‌های داده‌ی رابطه‌ای توزیعی است.

\section{انواع سامانه‌های ذخیره‌ای داده} \label{section:distributed-database-types}

\paragraph*{}
سامانه‌های ذخیره‌سازی داده‌ای که داده‌ها را به صورت رکورد ذخیره‌سازی می‌کنند، پایگاه‌داده نام دارند. پایگاه‌های داده به لحاظ خصوصیاتی که دارا هستند با یکدیگر تفاوت‌های بنیادین دارند. همینطور بدیهی است که در علوم مهندسی هنگامی که بحث انتخاب ابزار مورد استفاده برای پیشبرد و انجام کارها پیش می‌آید، انسان را مجاب به انتخاب میان ابزارهای مختلف با توجه به خصوصیاتی که از این تفاوت‌ها ناشی می‌شوند، می‌کند. از آنجایی که هدف از انجام این تحقیق، بهبود یکی از انواع سامانه‌های ذخیره‌ای داده است، بنابراین پس از معرفی انواع این سامانه‌های ذخیره‌ای، نوع خاصی که مد نظر است معرفی می‌گردد و سپس بهبود بر روی آن نوع صورت می‌پذیرد. در ادامه، سامانه‌های ذخیره‌ای داده‌ی مختلف بر اساس مشخصاتی که دارند معرفی می‌شوند و مورد مقایسه قرار می‌گیرند.

\paragraph*{}
انواع مختلف پایگاه‌های داده موجود عبارتند از:
\begin{enumerate}
\item
انباره‌های کلید-مقدار
\LTRfootnote{Key-Value Stores}:
ساده‌ترین نوع انباره‌ی داده است که در آن کل داده‌های ذخیره شده با کلیدها فهرست
\LTRfootnote{Index}
شده است. ورودی این انباره‌ی داده یک مقدار کلید است و خروجی آن، مقدار متناسب با آن خواهد بود.
\cite{cattell-2011}
مقدار ذخیره شده می‌تواند هر نوع مدل داده‌ای داشته باشد. (مانند عدد، رشته، آرایه یا شیء
\LTRfootnote{Object})
این انباره‌ها به طور معمول از تراکنش‌ها پشتیبانی نمی‌کنند یا اینکه افزودن پشتیباپایگاه‌دادهنی از تراکنش
\LTRfootnote{Transaction}
به این انباره‌ها سربار خیلی بالایی دارد.
\cite{stonebraker-2011}
این انباره‌های داده به منظور ذخیره‌سازی داده‌هایی است که به طور معمول از یک نوع هستند و این نوع شیء است.
\cite{cattell-2011}

\item
انباره‌های اسنادی
\LTRfootnote{Document Stores}:
به نسبت انباره‌های کلید-مقدار انواع داده‌های بیشتری را پشتیبانی می‌کند. در اینجا منظور از سند، هر نوع شیء بدون اشاره‌گر
\LTRfootnote{Pointerless}
است. مورد استفاده‌ی این انباره‌های داده، ذخیره‌ی اشیایی است که نوع آن‌ها یکی نیست و عملیاتی که بر روی انباره‌ی داده انجام می‌گیرد، بیشتر از نوع خواندن است و نه ذخیره‌سازی و ایجاد تغییر.
\cite{cattell-2011}
به مانند نوع قبل انباره داده‌ی ذکر شده، این نوع نیز از تراکنش پشتیبانی نمی‌کند.
\cite{stonebraker-2011}
 داده‌ی ذخیره شده در این نوع پایگاه‌های داده از نوع سند
\LTRfootnote{Document}
است.

\item
انباره‌های مبتنی بر رکوردهای توسعه‌پذیر (کشویی)
\LTRfootnote{Extensible Record Stores}:
این انباره‌ها جداول ذخیره‌ای داده‌ای هستند که داده‌ها را به صورت ردیفی (بر اساس رکورد) ذخیره می‌کنند. در این انباره‌ها، ستون‌های آن‌ها تعداد متغیری دارند و از آنجایی که هر ستون نشان‌دهنده‌ی یک خصوصیت برای شیء ذخیره شده در جدول است، این انباره‌های داده اشیایی با صفات
\LTRfootnote{Attributes}
مختلف در خود ذخیره می‌کنند. لذا مدل داده‌ای در این انباره‌ها تغییر می‌کند.
\cite{stonebraker-2011}
 داده‌ی ذخیره شده در این نوع پایگاه‌های داده از نوع رکوردهای توسعه‌پذیر است.

\item
پایگاه‌های داده‌ی
\lr{SQL}:
بر خلاف انباره‌های ذکر شده در بالا، این نوع از سامانه‌های ذخیره‌سازی داده مدل داده‌ای از قبل تعریف شده و مشخصی دارند و به طور پیش‌فرض پرس و جوها را به طور اتمیک
\LTRfootnote{Atomic}
اجرا می‌کنند. داده‌ی ذخیره شده در این نوع پایگاه‌های داده از نوع چندتایی
\LTRfootnote{Tupe}
است.
\cite{cattell-2011}
\end{enumerate}

\subsection{خصوصیات پایگاه‌های داده‌ی توزیعی} \label{subsection:database-features}

\subsubsection*{پشتیبانی از
\lr{ACID}
}
\paragraph*{}
پایگاه‌های داده‌ای که این خاصیت را دارا هستند، می‌بایست تمام چهار خاصیتی که در ادامه توضیح داده می‌شوند را دارا باشند که در ادامه یک به یک توضیح داده خواهند شد.
\cite{garcia-2008}

\begin{enumerate}
\item {تجزیه‌ناپذیری \LTRfootnote{Atomicity}}:
به معنای تجزیه‌ناپذیر بودن دستورات است، یعنی اگر چنانچه بخشی از یک تراکنش انجام نشود، پایگاه‌داده باید بدون تغییر باقی بماند و به حالتی بازگردد که قبل از ورود تراکنش در آن قرار داشت.

\item {هم‌خوانی \LTRfootnote{Consistency}}:
هر تراکنش باید پایگاه‌داده را از یک حالت معتبر به حالت معتبر دیگری ببرد. به این معنی که داده‌ها مطابق با قوانین تعریف شده‌ی پایگاه‌داده باشند.

\item {انزوا \LTRfootnote{Isolation}}:
یعنی هر تراکنش به نحوی اجرا شود که گویی مستقل از دیگر تراکنش‌ها و دستورات دیگر است. به این معنی که پایگاه‌داده می‌بایست این اطمینان را بدهد که دستورات در بدترین حالت پشت سر یکدیگر اجرا می‌شوند و تا پایان اجرای تراکنش در حال اجرا، تراکنش دیگری آغاز نمی‌گردد.

\item {پایایی \LTRfootnote{Durability}}:
براساس این خاصیت، تراکنش‌هایی که به مرحله انجام برسند اثرشان ماندنی است و هرگز به طور تصادفی از بین نخواهند رفت. تراکنشی که به طور کامل انجام شود طبق هیچ شرایطی (حتی شرایطی مانند قطع برق و حوادث غیر مترقبه) به حالت قبل از تراکنش باز نمی‌گردد.
\end{enumerate}

\subsubsection*{کارآیی \LTRfootnote{Performance}}
\paragraph*{}
در علوم کامپیوتر، کارآیی به معنی میزان انجام کار توسط سامانه‌ی کامپیوتری در واحد زمان است. به طور معمول و بر اساس فیلد مورد بررسی، کارآیی به یکی از این معانی تعبیر می‌شود: اول زمان پاسخ کوتاه برای تعدادی وظیفه
\LTRfootnote{Task}
ورودی، دوم توان عملیاتی
\LTRfootnote{Throughput}
(نرخ پردازش یا انجام وظایف) بالا و سوم پایین بودن بهره‌برداری
\LTRfootnote{Utilization}
از منابع سامانه. بر اساس سامانه‌ی مورد بررسی و یا محاسبات صورت گرفته برای اندازه‌گیری کارآیی، معیار اندازه‌گیری آن می‌تواند یکی از این دو مورد باشد.

\begin{latin}
	\begin{equation} \label{equation:performance}
		\text{Performance}
		\propto
		\frac{\text{Tasks}}{s} % 1
		\propto
		\text{Throughput} % 2
		\propto
		\frac{1}{\text{Resource Utilization}}
	\end{equation}
\end{latin}

\paragraph*{}
پس با توجه به معادله‌ی
\ref{equation:performance}
برای بالا بردن کارآیی سامانه‌های کامپیوتری یا باید توان محاسباتی سامانه بالا رود، یا زمان پاسخ آن کاهش یابد و یا از میزان بهره گیری از منابع سامانه کاسته شود.

\subsubsection*{تکرار \LTRfootnote{Replication}}
\paragraph*{}
تکرار به معنی ذخیره‌ی رونوشت
\LTRfootnote{Copy}
از اشیای ذخیره شده درون پایگاه‌داده، درون چندین پایگاه‌داده‌ی دیگر به منظور بالا بردن کارآیی و بهبود
\LTRfootnote{Improve}
دسترس‌پذیری
\LTRfootnote{Availability}
سامانه‌ی ذخیره‌ای داده است. دسترس‌پذیری از این جهت بالا می‌رود که اگر چنانچه یکی از اشیای مورد نیاز کاربر درون پایگاه‌داده‌ای وجود داشته باشد و آن پایگاه‌داده به هر دلیلی از شبکه خارج شود، پایگاه‌داده‌ی دیگری، یک رونوشت از آن شی‌ء را دارا است که منابع مورد نیاز کاربر در دسترس باقی بمانند.

\subsubsection*{مقیاس‌پذیری \LTRfootnote{Scalability}}
\paragraph*{}
برای هر سامانه‌ی اطلاعاتی، یک حجم داده‌ی ورودی و یک حجم داده‌ی خروجی تعریف می‌گردد که سامانه با انجام عملیاتی (مانند عملیات ذخیره‌سازی، پردازشی و غیره) ورودی‌ها را به خروجی‌ها تبدیل می‌کند. اگر چنانچه با بالا رفتن حجم ورودی‌های سامانه، در انجام عملیات توسط سامانه اخلال بوجود آمد، این به معنی نبودن خاصیت مقیاس‌پذیری در آن سامانه است.

\paragraph*{}
مقیاس‌پذیری در سامانه‌های کامیپوتری به دو دسته‌ی کلی زیر تقسیم می‌شوند:

\begin{enumerate}
	\item
مقیاس‌پذیری افقی:
به معنی امکان اجرای الگوریتم یا روش مورد نظر به صورت توزیعی و بر روی کامپیوتر‌های فیزیکی (یا مجازی) مختلف است. اگر چنانچه یک روش ارائه شده، امکان اجرای توزیعی بر روی کامپیوترهای مختلف را دارا نباشد، می‌توان گفت که آن روش مشخص به صورت افقی مقیاس‌پذیر نیست.
	\item
مقیاس‌پذیری عمودی:
به معنی امکان اجرای الگوریتم یا روش مورد نظر بر روی یک کامپیوتر و به صورت موازی
\LTRfootnote{Parallel}
تحت قالب نخ
\LTRfootnote{Thread}های
مختلف روی یک ماشین (پردازنده مشترک) است. اگر چنانچه یک الگوریتم را نتوان به صورت موازی اجرا کرد، در آن می‌توان گفت که آن روش به صورت عمودی، مقیاس‌پذیر نیست.

\end{enumerate}

\subsubsection*{قابلیت ارتجاعی \LTRfootnote{Elasticity}}
\paragraph*{}
قابلیت ارتجاعی میزان توانایی سامانه برای انطباق هرچه بیشتر با تغییرات در حجم کار است. به این معنی که مستقل از حجم درخواست‌های ورودی به سامانه، نسبت خروجی‌های موفق به این ورودی‌ها تغییر ننماید.
\cite{becker-2015}

\paragraph*{}
در رایانش ابری
\LTRfootnote{Cloud Computing - Utility Computing}
این اهمیت دارد که هر زمانی که منبعی مورد نیاز کاربران است، بتوان به کل سامانه منبعی را افزود و در عین حال در صورت آزاد شدن منابع در کل سامانه، بتوان منابع را آزاد نمود. به کارکرد اول افزودن مقیاس
\LTRfootnote{Scale-up}
و به کارکرد دوم کاهش مقیاس
\LTRfootnote{Scale-down}
تلقی می‌شود.
\cite{elastras-2010}
بسترهای ابری به منظور افزودن قابلیت تعمیر و نگهداری
\LTRfootnote{Maintenance}
به صورت لایه‌ای طراحی و پیاده‌سازی می‌شوند.
\cite{giriraj-2012}
به طور معمول در بسترهای ابری همه‌ی لایه‌ها قابلیت ارتجاعی را دارا هستند ولی لایه‌ی ذخیره‌ی داده از این موضوع مستثنی هستند. تلاش‌هایی در زمینه‌ی ارتجاعی کردن لایه‌ی ذخیره‌ی داده صورت گرفته است.
\cite{elasca-2013}

\paragraph*{}
به عنوان مثال در پایگاه‌داده‌ی مربوط به یک کتاب فروشی، در زمان‌هایی خریدها از سایت بسیار بیشتر خواهد بود نسبت به زمان‌های دیگر. پس سامانه‌ی پایگاه‌داده‌ای که به چنین سامانه‌ای سرویس ذخیره‌ای داده ارائه می‌دهد، باید بتواند که به طور مداوم (چه در زمان‌های بیشینه مصرف و چه در زمان‌های مصرف کم) تعداد درخواست‌های در ثانیه‌ی ثابتی قابلیت سرویس‌دهی داشته باشد.
\cite{becker-2015}


‌‌‌\subsection{دسته‌بندی پایگاه‌های داده بر اساس زبان پرس و جو} \label{subsection:databases-by-language}
\paragraph*{}
پایگاه‌های داده به منظور پاسخ‌دهی به پرس و جو
\LTRfootnote{Query}های
کاربران خود، زبان پرس و جویی برای دریافت و یا اعمال تغییر در داده‌های ذخیره شده درون خود، در اختیار کاربران قرار می‌دهند. پایگاه‌های داده به لحاظ زبان پرس و جوهای مورد نیاز برای ایجاد ارتباط با به دو دسته‌ی کلی مبتنی بر
\lr{SQL}
و
\lr{No-SQL}
تقسیم می‌گردند که هر یک خصوصیات منحصر به فردی دارند و برای کاربردهای متفاوت می‌توانند مورد استفاده قرار گیرند. پس بنابراین انتخاب این نوع پایگاه‌های داده برای استفاده، به طور مستقیم به کاربرد بستگی خواهد داشت و انتخاب آن با کاربر خواهد بود. تفاوت عمده‌ی این دو نوع پایگاه‌داده در مدل داده‌ای
\LTRfootnote{Data Model}
آن‌ها است. در پایگاه‌های داده‌ی
\lr{No-SQL}
مدل داده متغیر است.
\cite{cattell-2011}

‌‌‌\subsubsection*{پایگاه‌های داده مبتنی بر
\lr{No-SQL}
}
در ابتدای این بخش، چهار مورد از انواع پایگاه‌های داده‌ی رایج و مورد استفاده ذکر شدند. سه مورد اول از این چهار مورد، در دسته‌بندی پایگاه‌های داده‌ی مبتنی بر
\lr{No-SQL}
قرار می‌گیرند. این نوع پایگاه‌های داده، دارای خصوصیت نوع
\lr{ACID}
بسیار محدودی هستند و خواص مورد نیاز برای درخواست‌هایی که از نوع تراکنش هستند را شامل نمی‌شوند.
\cite{cattell-2011}

\paragraph*{}
این پایگاه‌های داده (یا برای این مورد بهتر است کلمه‌ی انباره‌ی داده اطلاق شود) به طور ذاتی این امکان را دارا هستند که داده‌های خود را در تکه
\LTRfootnote{Shard}هایی
توزیع کنند. یک یا چند تکه از پایگاه‌داده می‌توانند درون گره‌های ذخیره‌ای داده توزیع شوند. به همین دلیل است که به هنگام بروز رسانی داده‌ها، انباره‌ی داده می‌بایست داده‌ی بروز شده را در همه‌ی تکه‌های خود بروز رسانی نماید که این افزودن خاصیت پشتیبانی از تراکنش‌ها را محدود می‌کند.
\cite{stonebraker-2011}

\paragraph*{}
مشکل دیگری که این نوع پایگاه‌های داده با آن مواجه هستند این است که  امکان بروز خرابی
\LTRfootnote{Failure}
در گره‌های ذخیره‌ای وجود دارد و این نکته تا حدودی بر سختی نگهداری این نوع انباره‌های داده می‌افزاید. به طور کلی ایجاد همگامی
\LTRfootnote{Synchronization}
در گره‌ها هزینه‌ی زیادی دارد.
\cite{cattell-2011}

‌‌‌‌\subsubsection*{پایگاه‌های داده مبتنی بر
\lr{SQL}
}
\paragraph*{}
این نوع پایگاه‌داده دارای مدل‌داده‌ای از نوع رابطه‌ای
\LTRfootnote{Relational}
که این مدل داده‌ای برای تمامی چندتایی‌های ذخیره شده درون پایگاه‌داده ثابت است. همچنین شامل یک رابط زبان
\lr{SQL}
برای برقراری ارتباط با کاربر هستند و از خاصیت
\lr{ACID}
پشتیبانی می‌کنند. به تازگی انواعی از این پایگاه‌های داده را که به صورت توزیعی عمل می‌کنند طراحی کرده‌اند که از تراکنش‌ها نیز پشتیبانی می‌کنند.

‌‌‌\subsection{پایگاه‌های داده‌ی توزیعی و محلی} \label{subsection:databases-by-locality}
\paragraph*{}
پایگاه‌داده توزیعی عبارتست از تعدادی پایگاه‌داده‌ی توزیع شده در بستر یک شبکه‌ی کامپیوتری که از طریق بستر آن شبکه به یکدیگر متصل هستند. یک سامانه‌ی مدیریت پایگاه‌داده‌ی توزیعی
\LTRfootnote{Distributed Database Management System (DDBMS)}
سامانه‌ی نرم‌افزاری است که اجازه‌ی مدیریت پایگاه‌داده‌ی توزیع شده را به کاربر می‌دهد و خاصیت توزیعی پایگاه‌داده را از کاربر مخفی می‌نماید.
\cite{ozsu-2011}

\paragraph*{}
در گذشته به منظور بالا بردن مقیاس‌پذیری پایگاه‌های داده‌ی محلی، مجبور به توزیعی کردن بخش فیزیکی پایگاه‌داده شدند. به این معنی که تعدادی ماشین فیزیکی (معمولاً با یک مؤلفه‌ی مشترک) را به نحوی با یکدیگر شبکه کنند و اطلاعات را به منظور ذخیره‌سازی میان آن‌ها توزیع نمایند. پس دوره‌ای برای گذار از پایگاه‌های داده‌ی محلی به پایگاه‌های داده‌ی توزیعی وجود داشته است.
\cite{ozsu-2011}

\paragraph*{}
حال آنکه روش‌های مختلفی برای توزیعی کردن پایگاه‌های داده از گذشته تا به امروز ارائه، طراحی و پیاده‌سازی شده است. لازم به ذکر است که توزیعی کردن پایگاه‌داده به معنی ارائه‌ی یک معماری در سطح پایین‌تر از لایه‌ی نرم‌افزار (لایه‌ای که مدیریت
\LTRfootnote{Administration}
پایگاه‌داده در آن انجام می‌شود) است؛ چرا که در معماری لایه‌ای پایگاه‌های داده، توزیعی بودن باید از دید کاربر مخفی باشد.
\cite{sword-2014}

\paragraph*{}
پایگاه‌های داده توزیعی به لحاظ شیوه‌ی توزیعی بودن (یا نحوه‌ی قرار دادن داده‌ها در ماشین‌های فیزیکی) به سه دسته کلی تقسیم می‌شوند که عبارتند از:
\cite{cattell-2011}

\begin{enumerate}
\item
پایگاه‌های داده‌ی مشترک در حافظه
\LTRfootnote{Shared-Memory (SM)}:
اولین نسل پایگاه‌های داده‌ی توزیعی هستند که در آن‌ها به جای استفاده از بستر شبکه به منظور ایجاد ارتباط میان پایگاه‌های داده‌ی توزیع شده، بر روی یک کامپیوتر با حافظه‌ی مشترک استفاده می‌شده. در این نوع پایگاه‌داده‌ی توزیعی، مشکل این است که توزیع کردن داده‌ها میان گره‌ها باید توسط خود کاربر صورت بگیرد که برای انسان این کار وظیفه‌ی طاقت‌فرسا و عملاً نشدنی‌ای است.
\item
پایگاه‌های داده‌ی مشترک در دیسک
\LTRfootnote{Shared-Disk (SD)}:
راه دیگر برای ایجاد یک پایگاه‌داده توزیعی استفاده از چندین کامپیوتر به همراه دیسک خوشه‌ای
\LTRfootnote{Cluster Disk}
است که در آن تعدادی پردازنده به همراه حافظه‌های محلی خود به یک دیسک مشترک با ظرفیت زیاد متصل هستند. مشکلی که این نوع سامانه‌ها دارند، مشکل بودن ایجاد همگامی در کل سامانه است.
\item
پایگاه‌های داده بدون اشتراک
\LTRfootnote{Shared-Nothing}:
تا به اینجا این نوع معماری بهترین نوع معماری پایگاه‌داده توزیعی بشمار می‌رود و همانطور که از نام آن بر می‌آید، اتصال میان گره‌های ذخیره‌ای داده بدون به اشتراک گذاری دیسک و یا حافظه صورت می‌گیرد؛ بلکه اتصال از طریق بستر شبکه صورت می‌گیرد. این نوع معماری محدودیت‌های نام برده شده برای دو نوع قبل پایگاه‌های داده توزیعی را ندارد.
\end{enumerate}


‌‌\subsection{پایگاه‌های داده توزیعی رابطه‌ای} \label{subsection:relational-distributed-databases}
\paragraph*{}
همانطور که گفته شد، پشتیبانی از تراکنش نیازی است که از کاربردهای مهم و روزمره کاربران نشأت می‌گیرد و از پایگاه‌داده نیز این انتظار می‌رود که این نیاز را برآورده کند. همچنین به دلیل بالا رفتن حجم داده‌های مورد نیاز برای ذخیره در پایگاه‌های داده، رفتن به سمت پایگاه‌های داده‌ی توزیعی عملاً أمری اجتناب ناپذیر بشمار می‌رود. به همین دلیل نیاز به نوعی از پایگاه‌داده توزیعی احساس می‌شود که از مدل داده‌ای رابطه‌ی و زبان واسطه‌ی
\lr{SQL}
پشتیبانی کند. بنابراین پایگاه‌داده‌ای که قصد بهبود آن در این پایان‌نامه وجود دارد، از نوع توزیعی و رابطه‌ای است.

‌‌\section{انواع حجم کارها} \label{section:workload-types}

\paragraph*{}
در این بخش، کاربردهای مختلفی که می‌توان از سامانه‌های ذخیره‌ای اطلاعات انتظار داشت، یک به یک معرفی می‌شوند و میان این کاربردها، مورد مد نظر برای اعمال بهبود انتخاب و معرفی می‌گردد. طبیعی است که ابزاری که بیشترین مطابقت را با کاربرد انتخاب شده دارد برای بهبود انتخاب خواهد شد.

\paragraph*{}
اکثر کارکردهایی که کاربران امروزه از بستر شبکه انتظار دارند از بستر وب
\LTRfootnote{Web 2.0}
سرویس می‌گیرد. این نرم‌افزارها، به طور معمول به درخواست هایی از نوع
\LTRfootnote{OLTP}
پاسخ می‌دهند که در این نوع درخواست‌ها به طور معمول میلیون‌ها کاربر در حال خواندن یا بروزرسانی پایگاه‌داده هستند که این نوع کاربردها بر خلاف کاربردهای انباره‌های داده و سامانه‌های مدیریت پایگاه‌داده‌ی معمول هستند.
\cite{cattell-2011}
پس نوع پایگاه‌داده‌ی مورد استفاده وابسته به نوع کارکردی است که از نرم‌افزار مورد استفاده‌ی پایگاه‌داده انتظار می‌رود. به طور کلی هیچ راه حل ذخیره‌ای عام منظوره‌ای برای ذخیره‌ی داده‌های پایگاه‌های داده وجود ندارد.

\paragraph*{}
می‌دانیم که تراکنش درخواستی شامل واحدهای عملیاتی است که از سمت مشتری برای سرویس‌دهنده فرستاده می‌شود. مشتری انتظار دارد که یا تمام واحدهای عملیاتی درون تراکنش در سمت سرویس‌دهنده اجرا گردد یا اینکه هیچ یک از آن عملیات انجام و اعمال نشود. هر یک از این واحدهای عملیاتی عبارتند از تغییر در یکی از رکوردهای سامانه‌ی ذخیره‌ی داده‌ای سمت سرویس‌دهنده. واحدی که در سامانه‌ی ذخیره‌ی داده‌ای مورد تغییر قرار می‌گیرد رکورد پایگاه‌داده نام دارد.
\cite{elastras-2010}

\paragraph*{}
اما چیزی که در این پایان‌نامه اهمیت بیشتری دارد، تراکنش توزیعی است. در پایگاه‌های داده‌ی توزیعی، داده‌ها به نحوی در ماشین‌های فیزیکی مختلف توزیع شده‌اند.
\cite{schism-2010}
بدیهی است که تراکنش‌هایی که به پایگاه‌های داده وارد می‌شوند از نوع بازیابی اطلاعات و یا ایجاد تغییری در رکوردها هستند. هرگاه هر نوع تراکنشی که به پایگاه‌داده‌ی توزیعی وارد شود، باعث می‌شود که رکوردهای داده در ماشین‌های فیزیکی بازیابی یا به روز رسانی شوند و تعدادی ماشین فیزیکی درگیر پاسخگویی به تراکنش می‌شوند. اگر چنانچه برای یک تراکنش این تعداد از یک عدد بیشتر باشد، آن تراکنش یک تراکنش توزیعی
\LTRfootnote{Distributed Transaction}
نامیده می‌شود. هر چه تعداد ماشین‌های فیزیکی پاسخگو به تراکنش‌های وارد شده کمتر باشد، از نظر کارآیی و مصرف توان، سامانه وضعیت بهینه‌تری خواهد داشت.
\cite{kamal-2016}

\paragraph*{}
همانطور که در معادله‌ی
\ref{equation:performance}
نشان داده شد، به هر میزانی که تعداد گره‌های ذخیره‌ای داده‌ی کمتری درگیر پاسخ‌گویی به هر درخواست تراکنشی شود، کارآیی کلی سامانه نیز بیشتر می‌شود. پس مطلوب نهایی کاهش تعداد تراکنش‌های توزیعی در کل سامانه خواهد بود.

\paragraph*{}
در زمینه‌ی پایگاه‌های داده، حجم کار عبارتست از عملیاتی که توسط پایگاه‌داده انجام می‌گیرد. ولی به طور معمول منظور از حجم کار، نوع و خصوصیت درخواست‌های ورودی و تغییراتی است که در آن‌ها بوجود می‌آید. پس به طور کلی می‌توان به آن از نوع عملیات انجام گرفته در پایگاه‌داده تعبیر کرد. عملیات انجام گرفته می‌توانند از نوع بیشتر-نوشتنی و یا بیشتر-خواندی و همچنین از نظر پیچیدگی می‌تواند از نوع ساده یا پیچیده باشد.
\cite{cattell-2011}

\paragraph*{}
همانطور که گفته شد، تقسیم‌بندی حجم کارها را می‌توان از دو جنبه‌ی متفاوت انجام داد. اول از نظر میزان پیچیدگی پرس و جو‌های ارسالی به سامانه و دیگری از نظر نوع پرس و جوها - که می‌تواند متمرکز بر اعمال تغییرات بر روی پایگاه‌داده باشند و یا تنها آن را می‌خوانند. در شکل
\ref{image:workload-categorization}
تقسیم‌بندی حجم کارها بر این دو طیف مشاهده می‌شوند. همچنین در این تصویر، محل قرار گرفتن نرم‌افرازهای معمول مورد استفاده در دنیا به طور مفهومی قرار گرفته است. مشاهده می‌شود که کاربردهای شبکه‌های اجتماعی و تراکنشی در دسته‌ی عملیات ساده
\LTRfootnote{Simple Operations}
قرار می‌گیرند.
\cite{cattell-2011}


\begin{figure}[h]
	\includegraphics[scale=.85]{workload-categorization}
	\caption{دسته‌بندی حجم کارها در دو طیف بر اساس عملیات آن‌ها}
	\label{image:workload-categorization}
\end{figure}

\paragraph*{}
توزیع داده‌های ذخیره شده در پایگاه‌داده توزیعی، می‌بایست به نحوی باشد که با توجه به استفاده‌ای که از داده می‌شود، بهترین کارآیی را داشته باشد. پس بنابراین در نحوه‌ی توزیع داده، حجم کاری که داده توسط آن مورد استفاده قرار می‌گیرد نیز باید لحاظ گردد.
\cite{kamal-2016}
به این نوع توزیع داده، توزیع داده مبتنی بر حجم کار
\LTRfootnote{Workload-Aware}
گفته می‌شود.

\paragraph*{}
حال با توجه به مطالب فوق، می‌توان گفت که راه حل‌های ذخیره‌ای داده در پایگاه‌های داده، می‌توانند به دو دسته ی کلی تقسیم شوند:

\begin{enumerate}
	\item
پایگاه‌های داده‌ای که بسیار کم بروزرسانی می‌شوند و در اکثر موارد فقط-خواندنی هستند.
	\item
پایگاه‌های داده‌ای که خواندنی-نوشتنی هستند و تراکنش‌های به نسبت پیچیده‌ای بر روی آن‌ها در هر لحظه می‌تواند اعمال کرد.
\end{enumerate}

‌\section{بخش‌بندی پایگاه داده‌ی رابطه‌ای بدون اشتراک برای حجم کارهای تراکنشی}

\paragraph*{}
در این بخش هدف این است که نوع سامانه، الگوریتم و تابع هدفی که قصد بهبود آن وجود دارد به طور کامل و جزئی معرفی گردد.

% انواع مدل‌ها...
\subsection{انواع مدل‌های بخش‌بندی}

\paragraph*{}
به منظور بخش‌بندی داده‌های توزیعی در ماشین‌های فیزیکی، ابتدا داده‌ها می‌بایست به نحوی مدل شوند، بخش‌بندی بر روی مدل اعمال شده و در انتها بر اساس مدل بخش‌بندی شده، توزیع بر روی داده‌ها صورت می‌گیرد. در سال‌های اخیر به منظور بخش‌بندی داده‌های پایگاه‌های داده برای توزیعی کردن آن‌ها، داده‌ها را با گراف
\LTRfootnote{Graph}،
هایپرگراف
\LTRfootnote{Hypergraph}
و یا با روش‌های تلفیقی
\LTRfootnote{Hybrid}
مدل می‌کنند.
\cite{kamal-2016}

% تبیین مدل...
\paragraph*{}
معمولاً  گره‌ها نمایش‌دهنده‌ی واحد تعریف شده‌ی داده‌ها (مثلاً اشیاء) است و یال میان گره‌ها، رابطه و همبستگی میان این داده‌ها را نشان می‌دهد. همبستگی میاد هر دو واحد داده عبارتست از تعداد درخواست‌های از نوع تراکنشی که آن دو واحد داده در آن با یکدیگر مورد دسترسی قرار می‌گیرند. در برخی کارها به منظور بخش‌بندی داده‌ها، مدل گراف تولید شده را به منظور کاهش سربار بخش‌بندی ابتدا به روش‌هایی فشرده‌سازی می‌کنند و سپس آن را بخش‌بندی می‌کنند که این کار کیفیت بخش‌بندی را به شدت پایین می‌آورد.
\cite{schism-2010}

% ایراد مدل گراف...
\paragraph*{}
در مدل گراف، اگر چنانچه دو گره با یکدیگر درون حداقل یک تراکنش باشند، میان آن‌ها یالی قرار می‌گیرد که تا حدی همبستگی میان این دو گره را باز می‌تابد. این در حالی است که چنانچه دو گره در تعداد تراکنش بیشتری با یکدیگر قرار گیرند، لازم است که با احتمال بیشتر در یک بخش قرار گیرند که این مسئله در بخش‌بندی لحاظ نمی‌گردد. تابع هدفی که در مدل گراف تعریف می‌شود همان تعداد زوج اشیایی که حداقل در یک تراکنش توزیعی قرار دارند.

% رفع مشکل مدل گراف با هایپرگراف...
\paragraph*{}
این مشکل با استفاده از مدل هایپرگراف مرتفع می‌گردد چرا که هر هایپریال در مدل هایپر گراف استفاده شده، نمایانگر یک تراکنش توزیعی است. به این صورت که به ازای هر تراکنش در حجم کار، میان گره‌هایی (که همان مدل شده‌ی اشیاء هستند) که درون یک تراکنش قرار می‌گیرند، یک هایپریال ترسیم می‌شود. به این صورت می‌توان گفت که با بخش‌بندی هایپرگراف، دقیقاً تعداد تراکنش‌های توزیعی کمینه می‌شوند و نه تعداد زوج اشیایی که حداقل در یک تراکنش توزیعی قرار دارند.

% مشکل مدل هایپرگراف...
\paragraph*{}
پس با بهره‌گیری از مدل هایپرگراف، می‌توان مطمئن شد که تابع هدف به طور صحیح تعریف می‌گردد. ولی استفاده از این مدل برای بخش‌بندی، سربار بسیار زیادی را بر بخش‌بندی کننده
\LTRfootnote{Partitioner}
متحمل می‌کند.
\cite{kamal-2016}
به همین جهت، در مدل‌های ارائه شده، معمولاً مدل هایپرگراف در نظر گرفته شده را با روش‌های اکتشافی
\LTRfootnote{Heuristic}
به گراف تبدیل می‌کنند و بخش‌بندی را بر روی گراف بدست آمده اعمال می‌کنند که در این صورت، روش مدل‌سازی به روش تلفیقی تلفیقی مبدل می‌گردد.
\cite{sword-2013}


% انواع بخش‌بندی پایگاه‌های داده‌ی توزیعی...
\subsection{انواع روش‌های بخش‌بندی پایگاه‌داده‌های رابطه‌ای}

\paragraph*{}
در پایگاه‌های داده‌ی توزیعی رابطه‌ای، انواع بخش‌بندی عبارتند از:

\begin{enumerate}
	\item
بخش‌بندی افقی: به این معنی است جداسازی و توزیع جداول پایگاه‌داده بر مبنای جداسازی ردیف‌های جداول پایگاه‌داده صورت بگیرد. هر چند بسته به مدل‌سازی داده‌های درون پایگاه‌داده این کار روش‌های متفاوتی دارد؛ ولی روش معمول برای بخش‌بندی افقی پایگاه‌های داده‌ی توزیعی به این صورت است که هر ردیف از جداول پایگاه‌داده را معادل یک گره از گراف مدل شده در نظر می‌گیرند. این روش جایی کاربرد دارد که در آن تعداد ردیف‌های جداول پایگاه‌داده مرتبه‌ی بسیار بیشتری نسبت به ستون‌های آن دارند.
	\item
بخش‌بندی عمودی: به معنی جداسازی و توزیع جداول پایگاه‌های داده بر مبنای ستون‌های جداول است که در آن هر ستون معادل یک المان (یا معمولاً گره‌ی گراف) از مدل بخش‌بندی در نظر گرفته می‌شود. کاربرد این مورد در زمان‌هایی است که ستون‌های 
\end{enumerate}

% انواع روش‌های بخش‌بندی در مدل گراف...
\subsection{انواع مدل‌های بخش‌بندی در مدل گراف}

\paragraph{}
دو روش کلی برای بخش‌بندی گراف وجود دارد که عبارتند از:
\cite{rahimian-2014}

\begin{enumerate}
	\item
بخش‌بندی برش یال:
	\item
بخش‌بندی برش گره:
\end{enumerate}

% تابع هدف...
\subsection{تعریف تابع هدف بخش‌بندی}

\paragraph*{}
همانگونه که گفته شد، هدف از بهبود بخش‌بندی، بیشینه کردن کارآیی است و با توجه به معادله‌ی
\ref{equation:performance}
برای بهبود کارآیی باید تعداد ماشین‌های فیزیکی درگیر در هر دسترسی به سامانه، کمینه گردد که این هنگامی محقق می‌گردد که تعداد تراکنش‌های توزیعی کمینه گردد. پس تابع هدف مسئله‌ی بخش‌بندی داده‌ها، کمینه کردن تعداد تراکنش‌های توزیعی است.

% لزوم تکرار...
\paragraph*{}
یکی از سرویس‌های پیشفرض و مورد نیاز برای سامانه‌هایی که سرویس‌هایی از نوع ابر ارائه می‌دهند، دسترس‌پذیری است که در سامانه‌های مبتنی بر ابر، این سرویس توسط تکرار فراهم می‌آید.
\cite{velte-2010}
در پایگاه‌های داده‌ی توزیعی نیز این سرویس به همین طریق برای کاربران فراهم می‌آید. تکرار در این نوع سامانه‌ها در موارد عملی پیاده‌سازی شده، سطوح مختلفی دارد. در مواردی این تکرار در سطح چندتایی‌ها
(\cite{schism-2010}، \cite{sword-2013})
و در برخی در سطح ماشین‌های فیزیکی ذخیره‌ای داده
(\cite{kamal-2016})
صورت می‌گیرد.

% بخش‌بندی مجدد...
\subsection{لزوم بخش‌بندی مجدد}
\paragraph*{}
در پایگاه‌های داده‌ی توزیع شده، هنگام بارگزاری سامانه ذخیره‌ای داده، داده‌های اولیه در ابتدا بخش‌بندی می‌شوند و هر یک به ماشین فیزیکی مربوط به خود وارد و ذخیره خواهد شد. پس از گذشت زمان و تغییر در مشخصه‌های آماری حجم کار بخش‌بندی اولیه که برای داده‌های اولیه مناسب بوده است، کارآمدی خود را از دست می‌دهد و نیاز به بخش‌بندی مجدد برای بازیابی کارآمدی اولیه دارد. از آنجایی که بخش‌بندی مجدد کل داده عملی با سربار زیاد و غیر کارآمد است، پس راه حل، بهبود بخش‌بندی قبل به نظر می‌رسد. به این صورت که بخش‌بندی با نظر به مهاجرت داده‌ها صورت می‌گیرد. پس این عمل به دلیل تقابل با تأثیرات منفی تغییرات در مشخصه‌های آماری حجم کار صورت می‌گیرد. هدف از این کار، پیدا کردن زوج‌های واحد داده‌ی ذخیره شده به منظور مهاجرت میان ماشین‌های فیزیکی است که باعث کاهش فرکانس تراکنش‌های توزیعی می‌گردد.
\cite{sword-2013}

% reactive and proactive repartitioning...
\paragraph*{}
به منظور انطباق توزیع داده‌ها در پایگاه‌داده توزیعی با حجم کاری که با داده‌ها سر و کار دارد، روش‌هایی وجود دارند که به دو دسته‌ی کلی تقسیم می‌گردند.
\cite{kamal-2016}
این دو نوع عبارتند از:

\begin{enumerate}
	\item
روش‌های واکنشی
\LTRfootnote{Reactive}:
در این روش‌ها الگوریتم به طور دوره‌ای با دوره‌ی تناوب مشخصی اقدام به شناسایی تغییرات (عمدتاً منفی) رخ داده در حجم کار و بررسی تأثیرات آن بر روی کیفیت ارائه‌ی سرویس سامانه می‌نماید و با توجه به این مؤلفه‌ها، تغییراتی در پایگاه‌داده‌ی توزیعی و نحوه‌ی توزیع داده‌ها در آن اعمال می‌نماید که مؤلفه‌های کیفیتی مورد نظر به میزان مطلوب گذشته باز گردد. حُسن این نوع روش‌ها در این است که تأثیرات منفی کاهش کیفیت ارائه‌ی سرویس سامانه از دید کاربر دیده نمی‌شود و در عوض سربار زیاد بررسی کاهش کیفیت در این میان خودنمایی می‌کند.

	\item
روش‌های فعالانه
\LTRfootnote{Proactive}:
در این روش‌ها تا زمانی که تأثیرات منفی تغییر در حجم کار رؤیت نشود، تغییرات در توزیع داده‌ها صورت نمی‌گیرد. حُسن این روش در این نکته است که سربار پردازشی بررسی متناوب سامانه، دیگر بر دوش آن قرار ندارد و تا زمانی که نیاز به تغییر احساس نشد، پردازشی در این زمینه صورت نمی‌پذیرد.

\end{enumerate}


‌\section{معرفی کارهای مرتبط}

\paragraph*{}
در این بخش، سابقه‌ی تحقیق یک به یک معرفی می‌گردند و برای هر مورد به ترتیب مدل و معماری پیشنهادی، بهبود ارائه شده و همچنین ایرادات روش پیشنهاد شده ذکر می‌شود.

% schism 2010...
% مدل و معماری
\paragraph*{}
در
\cite{curino-2011}
یک روش مبتنی بر حجم کار برای بخش‌بندی و همچنین تکرار در پایگاه‌های داده توزیعی رابطه‌ای ارائه شده است که در آن کل پایگاه‌داده‌ی رابطه‌ای برای سادگی به یک جدول با تعداد ستون‌های زیاد و ردیف‌های بسیار زیاد تعبیر می‌شود. برای ایجاد مدل، بخش‌بندی افقی بر روی جدول بزرگ صورت می‌گیرد و هر یک از بخش‌های تولید شده یک گره از یک گراف در نظر گرفته می‌شود. پس هر یک از ردیف‌های جدول پایگاه‌داده به یک گره مدل می‌شود. همچنین یال میان هر دو گره از گراف نشان می‌دهد که ردیف‌های متناظر آن‌ها در پایگاه‌داده درون یک تراکنش قرار می‌گیرند.
\cite{schism-2010}

\paragraph*{}
پس از ایجاد مدل، بر روی گراف بخش‌بندی با الگوریتم
\lr{Metis}
\cite{metis-1996}
انجام می‌شود که یک الگوریتم بخش‌بندی گراف از خانواده‌ی روش‌های چند مرحله‌ای است. پس از ایجاد بخش‌ها در گره‌های گراف، هر یک از بخش‌ها درون یک ماشین فیزیکی نگاشت
\LTRfootnote{Map}
می‌شوند و اطلاعات این نگاشت درون یک جدول به صورت ریزدانه
\LTRfootnote{Fine-Grained}
ذخیره می‌گردد.

% مشکلات...
\paragraph*{}
در مدل این روش به منظور کاهش سربار بخش‌بندی گراف، از فشرده‌سازی استفاده می‌شود که همانطور که گفته شد، این کیفیت بخش‌بندی را به میزان زیادی پایین می‌آورد. همچنین در این روش، تغییرات حجم کار در توزیع داده‌ها هیچ گونه تأثیری نمی‌گذارند و به مرور زمان بخش‌بندی صورت گرفته برای حجم کار جهش‌یافته
\LTRfootnote{Evolved}
ناکارآمد می‌شود.
\cite{kamal-2016}

% sword 2013...
\paragraph*{}
در
\cite{sword-2013}
مدل ارائه شده، مدل هایپرگراف است که به منظور کاهش سربار اعمال الگوریتم بخش‌بندی بر روی هایپرگراف از روش دو مرحله‌ای استفاده می‌کند. در این روش با روش‌های چکیده‌سازی
\LTRfootnote{Hashing}
فشرده می‌شود و پس از آن الگوریتم بر روی هایپرگراف فشرده شده انجام می‌شود. همانطور که گفته شد، این مدل به هدف نهایی نزدیک‌تر است.

% مشکلات...
\paragraph*{}
به منظور رفع مشکل تغییر در حجم کار، این روش در بازه‌های زمانی ثابت، اقدام به شناسایی واحدهای داده‌ای برای مهاجرت بین سرویس‌دهنده‌های فیزیکی می‌کند که با حجم کار تغییر یافته سازگار شود که این روش واکنشی برای انطباق توزیع داده‌ها با حجم کار است. هر چند که این روش لزوماً تعداد تراکنش‌های توزیعی را کمینه نمی‌کند.

% Workload-aware incremental repartitioning of shared-nothing distributed databases for scalable OLTP applications 2016

% مدل ...
\paragraph*{}
مدل و معماری پیشنهادی در
\cite{kamal-2016}
در حال حاضر بهترین از نوع خود
\LTRfootnote{State of Art}
است. در این روش نیز مانند تمام روش‌های نام برده شده، کل پایگاه‌داده به منظور سادگی به یک جدول بزرگ مدل شده به نحوی که هر یک از ردیف‌های جدول نمایانگر (مدل کننده‌ی) یک شیء است که با یک شناسه‌ی منحصر به فرد
\LTRfootnote{Unique Identifier}
مشخص می‌شود. هر یک از این ردیف‌ها به یک گره از گراف مدل نگاشت می‌شود و یال میان هر یک از گره‌های گراف نمایانگر این است که ردیف متناظر آن‌ها حداقل درون یک تراکنش با یکدیگر مورد دسترسی قرار می‌گیرند.

% معماری ...
\paragraph*{}
معماری پیشنهادی برای چنین سامانه‌ی پایگاه‌دادهای در این روش به این صورت است که تراکنش‌های مورد نیاز کاربران از طریق لایه‌ی نرم‌افزاری
\LTRfootnote{Application Layer}
وارد سامانه می‌شوند و اطلاعاتی از آن به ماشین فیزیکی می‌رود که وظیفه‌ی بخش‌بندی گراف مدل شده را بر عهده دارد. این ماشین با ایجاد مدل گراف با توجه به تراکنش‌های ورودی، آن را بخش‌بندی می‌کند و پس از آن، اطلاعات بخش‌بندی مورد نظر را به ماشین‌های ذخیره‌کننده‌ی داده ارسال می‌نماید تا مهاجرت داده‌های مشخص شده انجام پذیرد.

% repartitioning...
\paragraph*{}
بخش‌بندی مجدد به صورت دوره‌ای بر روی داده‌ها صورت می‌گیرد و برای انطباق با حجم کار تغییر یافته شده، این الگوریتم یک پنجره‌ی حساس
\LTRfootnote{Sensitive Window}
بر روی
\lr{Log}های
دریافتی از طرف لایه‌ی نرم‌افزار تعریف کرده است که بخش‌بندی را با داده‌های دریافتی از آن پنجره انجام می‌دهد و باقی
\lr{Log}هایی
که خارج آن پنجره هستند، در بخش‌بندی تأثیری ندارند.

% مشکلات ...
% مشکل مدل ...
\paragraph*{}
همچنین در این مدل‌سازی، به ازای هر دو شیء درون یک تراکنش رابطه‌ای با وزن ثابت فرض شده است. پس مدل ذکر شده، میان همبستگی دو شیء با ۱۰۰ تراکنش مشترک و دو شیء با یک تراکنش مشترک تفاوتی قائل نمی‌شود. بنابراین تابع هدف تعریف شده برای این روش بخش‌بندی به درستی تعریف نشده است و تعداد تراکنش‌های توزیعی را به طور مستقیم کاهش نمی‌دهد.

% مشکل معماری ...
\paragraph*{}
همانطور که ذکر شد در این معماری، ماشین فیزیکی که وظیفه‌ی بخش‌بندی داده‌ها را بر عهده دارد به صورت متمرکز این کار را انجام می‌دهد که این خود برای سیستم یک نقطه‌ی شکست
\LTRfootnote{Breaking Point}
از نظر توزیع‌پذیری به شمار می‌رود. از طرفی با بالا رفتن ترافیک مورد استفاده‌ی سامانه و بروزرسانی‌های پیاپی، چاره‌ای به جز ارتقای این جزء غیر توزیعی وجود نخواهد داشت. پس طی مرور زمان نیاز است که این بخش از سامانه از نظر مشخصات فیزیکی ارتقاع یابد و این قضیه با خاصیت کشسانی در رایانش ابری در تضاد است؛ چرا که در سامانه‌های مبتنی بر ابر، این لازمه وجود دارد که با کاهش ترافیک سامانه بتوان مقادیری از منابع (از جمله منابع سخت‌افزاری) استفاده شده در سامانه را آزاد نمود و با افزایش آن بتوان منابعی به آن افزود.

\paragraph*{}
در حال حاضر مدل‌هایی وجود دارند که مشکل خاصیت کشسانی مورد نیاز رایانش ابری را تا حدی رفع نموده باشند 
(\cite{elastras-2010}
و
\cite{elasca-2013})
ولی در این روش‌ها مشکل عدم وجود بخش‌بندی مجدد مبتنی بر حجم کار و همچنین مشکل رایج عدم رعایت تابع هدف مدل بخش‌بندی خودنمایی می‌کند.

% توزیع بار...
\paragraph*{}
در روش
\cite{kamal-2016}،
مدل گراف تولید شده طی الگوریتم بخش‌بندی استفاده شده
\cite{metis-1996}،
گراف به
\lr{k}
عدد بخش منطقی
\LTRfootnote{Logical Partitions}
بخش‌بندی می‌شود که این تعداد بزرگ‌تر از تعداد گره‌های ذخیره‌ای داده است. علت این کار، ریزدانگی بیشتر تخصیص بخش‌ها به ماشین‌های فیزیکی به منظور کاهش خطای توزیع بار
\LTRfootnote{Load Balancing}
عنوان شده است. در واقع در این روش، توزیع باری مناسب در نظر گرفته شده است که رابطه‌ی
\ref{equation:good-load-balance}
در آن برقرار باشد.
\begin{latin}
	\begin{equation} \label{equation:good-load-balance}
		k \times \frac{\text{max}( w_{V_{i}} )}{w_{v}}
		\mapsto
		1
	\end{equation}
\end{latin}

\paragraph*{}
به این معنی که مقدار این رابطه هر چه بیشتر به عدد یک نزدیک‌تر باشد، نتیجه‌ی توزیع بار بهتر خواهد بود. لازم به ذکر است که در این رابطه مقدار
\lr{ $ w_{V_{i}} $ }
برابر است با مجموع وزن‌های گره‌ها در بخش
\lr{i}
همچنین
\lr{ $ w_{V} $ }
مجموع وزن‌های تمام گره‌های گراف و مقدار
\lr{k}
تعداد کل بخش‌هایی که داده‌ها در آن‌ها افراز شده‌اند.

% elastras 2010...
\paragraph*{}
از روش‌هایی که تمرکز کار آن‌ها بر روی افزودن خاصیت کشسان به پایگاه‌های داده‌ی توزیعی است می‌توان به
\cite{elastras-2010}
اشاره کرد. در این روش بخش‌بندی در سطح شمای پایگاه‌داده
\LTRfootnote{Database Schema}
انجام شده است؛ به این صورت که شمای اصلی پایگاه‌داده به عنوان ریشه‌ی یک ساختار داده‌ی درخت در نظر گرفته می‌شود و با بخش‌بندی عمودی پایگاه‌داده، برای هر گره از درخت، فرزندی ایجاد می‌کند. در گذر زمان و با افزوده شدن داده به پایگاه‌داده، نیاز به بزرگ شدن اندازه است، پس در عمق درخت پیشروی صورت می‌گیرد و شمای مورد استفاده توزیعی‌تر می‌گردد. اگر حجم داده‌های ذخیره شده در پایگاه‌داده کاهش یافت، با کم کردن عمق شمای مورد استفاده، از اندازه‌ی سامانه کاسته می‌شود و به این صورت خاصیت کشسانی به سامانه داده می‌شود.

\paragraph*{}
ضعف این روش در این مورد است که ساختار کلی شمای پایگاه‌داده در هر تغییر باید تغییر کند و تغییر در شمای پایگاه‌داده بسیار زمان‌بر و پُر محاسبه است. پس بنابراین در بخش‌بندی مجدد نمی‌توان از آن استفاده نمود و بخش‌بندی صورت گرفته ایستا
\LTRfootnote{Static}
است و قابل استفاده برای حجم کارهای مورد نظر این پایان‌نامه نخواهد بود.

% elasca 2013...
% sword 2014...
