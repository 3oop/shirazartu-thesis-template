% the introduction section...

\chapter{سابقه‌ی تحقیق} \label{chapter:related-works}

\paragraph*{}

با ورود به عصر اطلاعات و پیشرفت بیوانفورماتیک و توالی یابی ژنتیکی، موزه های تنوع زیستی با داشتن گنجینه وسیعی و متنوعی از گونه های زیستی بسیار کمک کننده هستند.

مک لین و دیگران (2016) به بررسی این موضوع پرداختند کمک به مجموعه های تاریخ طبیعی را به سمت تحقیق ببرد. آنها مامولوژی را به عنوان یک مثال، و مشارکت آن را تجزیه و تحلیل کرد مجموعه های موزه ای که برای تحقیق در مقالات ساخته شده اند منتشر شده در مجله \lr{Mammalogy} در طول 2005-14. آنها دریافتند که مجموعه پستانداران به تحقیقات در پنج حوزه گسترده کمک کرد و آن 25 درصد از کل مقالات مجله \lr{Mammalogy} به نحوی از مجموعه های تاریخ طبیعی استفاده کردند. نتایج آنها «این تاریخ طبیعی را نشان می دهد مجموعه‌ها زیرساخت های حیاتی را پشتیبانی می کنند تعداد قابل توجهی از انتشارات تحقیقاتی سالانه آنها همچنین نشان می دهد که استفاده از تاریخی است نمونه‌ها علاوه بر کوپن مداوم مجموعه [نمونه] یک رویکرد جدایی ناپذیر است به بسیاری از سؤالات تحقیقاتی در پستاندار شناسی».
پنج حوزه تحقیقاتی گسترده ای که موزه مجموعه‌های پشتیبانی شده عبارتند از: 
\begin{itemize}
    \item     سیستماتیک و جغرافیای زیستی، به عنوان اولیه آرشیو داده های جغرافیایی زیستی نشان می دهد تنوع زیستی و تغییرات در طول زمان
    \item    ژنومیک، بررسی پاسخ های ژنتیکی به تغییرات محیطی، به عنوان مثال از دست دادن تنوع ژنتیکی در سنجاب های آلپ همانطور که محدوده آنها در پاسخ به تغییر می کند تغییرات آب و هوایی 
    \item    مورفولوژی و ریخت سنجی که کاربردهای متعددی در سیستماتیک دارند و مطالعه پاسخ های بیولوژیکی به تغییر محیطی
    \item    اکولوژی ایزوتوپ پایدار، که می تواند مورد استفاده قرار گیرد برای کشف تغییرات در اکولوژی در طول زمان، رفتار مهاجرتی و ایجاد تنوع تولید مثل و زمستان گذرانی از حیوانات 
    \item    انگل‌ها و پاتوژن ها، کمک به درک تغییرات در توزیع انگل‌ها و رویدادهای تغییر میزبان در پرتو تغییرات محیطی، به ویژه آب و هوا تغییر و معرفی
\end{itemize}

آنها به اهمیت کلیدی آن اشاره کردند شیوه های علمی-کیوریتوری زیر: • جمع آوری نمونه کوپن یک ضرورت است برای سودمندی مداوم مجموعه‌ها • نمونه‌ها باید به طور مؤثری تنظیم شوند و در اشکال مختلف نگهداری می شود حفظ ارتباط با محیط زیست و سایر اطلاعات زیست محیطی تا آنجا که امکان پذیر است، به ویژه ارجاع جغرافیایی. • داده‌ها باید به خوبی مدیریت شوند و قابل کشف، و به طور گسترده در دسترس است از طریق به عنوان مثال \lr{GBIF}، \lr{GenBank} (ژنتیکی)، \lr{Morphbank} (مورفولوژیکی)، \lr{ViPr} (ویروس شناسی) و ماشین زمانی (سری زمانی)


\section{نور و شرایط محیطی}

برای داشتن بهره وری لازم از یک محیط و رعایت اصول آرگونومی در آن لازم است که یک روشنایی مناسب در محیط ایجاد کنیم.
نور طبیعی اعم از مرئی و نامرئی از نگرانی های عمده ی موزه‌داران است. نور طبیعی کلیه فرکانس های انرژی الکترومغناطیسی یا تابشی است. آن چه میبینیم تنها بخش کوچکی از کل طیف نور است که زیانبار ترین شکل تابش نیست. این بخش نامرئی نور است که برای اشیا ضرر دارد.

جدول \ref{tbl2.1} راهنمایی است جهت نور پردازی اشیا به منظور اجتناب از تاثیر زیانبار اشعه ی فرابنفش

\begin{table}[h!]
    \label{tbl2.2}
    \centering
    \begin{tabular}{|c|c|}
        \hline
        شدت نور بر حسب (\lr{FC}\RTLfootnote{فوت شمع (\lr{FC}) واحد سنجش نور است، هر فوت شمع برابر است با ۱۰ لوکس}) & اشیا حساس به فرابنفش \\
        \hline
        \lr{FC} 15               & \makecell{نقاشی‌های رنگ روغن و رنگ‌های لعابی، \\ \textbf{چرم طبیعی، شاخ استخوان، لاک، عاج} }                                                           \\
        \lr{FC} 5                & \makecell{بافته‌ها، پارچه‌ها، دیوارکوب، مخطوطات، کتاب، \\ تمبر، گواش، آبرنگ، چرم رنگ آمیزی شده، \\ \textbf{نمونه‌های گیاهی، پوست، خز، پر، حشرات}  }    \\
        \lr{FC} +40              & فلز، سنگ، آبگینه، سفالینه، جواهر، مینا و چوب طبیعی                                                                                     \\
        \hline
    \end{tabular}
    \caption{حساسیت اشیا مختلف نسبت به نور فرابنش}
\end{table}

بعضی مواد نسب به اشعه ی فرابنفش فوق العاده حساس هستند. موادی چون مو، پر، چرم، ابریشم، عاج و برخی رنگ‌ها در برابر تاثیر نور آسیب پذیرند. در میان منابع نورپردازی عمومی نور فلورسنت بیشتر اشعه ی فرابنفش را تولید میکند و نور لامپ های حرارتی بیشترین حرارت را باعث می شوند. روش های ساده محدود ساختن آسیب ناشی از تابش عبارتند از :
\begin{itemize}
    \item    نصب صافی (فیلتر) روی لامپ عای فلورسنت
    \item    رعایت فاصله و تهویه در موارد استفاده از نورپردازی با لامپ های حرارتی
\end{itemize}

رطوبت، دما و نور باعث فرسایش اشیا میگردند. دمای ۱۵ درجه سانتیگراد و رطوبت نسبی ۶۰٪ برای اکثر محیط های موزه ای مناسب است. رطوبت نسبی نباید کم و زیاد گردد و این مسئله با اشیا ارتباط مستقیم دارد ولی در کل رطوبت نسبی ۵۰٪ تا ۶۰٪ توصیه می شود

از انجا که نمونه های موجود در موزه بسیار حساس هستند به صورت مرتب باید دما و رطوبت بررسی شوند.

یک موزه بسته به معماری آن و نوع نمایشگاه می تواند به طرق مختلف نورپردازی شود
در موزه ی گوگنهایم اثر فرانک لوید رایت در نیویورک، آشکار است که نور در بافت موزه نه تنها یک نقش عملکردی را ایفا می کند، یعنی به شما اجازه «دیدن» می دهد، بلکه یک نقش دیگر نیز دارد عملکرد زیبایی شناختی که خود ساختار معماری را تقویت می کند. با وجود این چشم انداز بسیار متغیر، برخی از جنبه‌های نورپردازی موزه‌ها به نیازها و عملکردهای مشترک پاسخ می‌دهند. در موزه ها، عمل دیدن غالب است. بینایی به حس اساسی تبدیل می شود تجربه زیبایی شناختی به طوری که محیط نمایشگاه تمایل به بیرون راندن هر حضور دیگری که حواس را پرت می کند دارد.



اولین موزه های عمومی قرن هجدهم با نور طبیعی روشن شدند. به طور مشخص، این یک ضرورت بود زیرا منابع نور مصنوعی موجود هنوز بسیار ابتدایی بودند دسترسی فراوان به آن‌ها مشکل بود همجنین برای حفاظت از اشیای با ارزش موجود در موزه‌ها بسیار خطرناک بودند. 
طراحان موزه‌ها از نور روز استفاده می‌کنند زیرا انسان‌ها با طبیعت ارتباط دارند. "نور طبیعی را می توان برای نمایش و جان بخشیدن به طراحی استفاده کرد.
طرح‌های موزه‌ها از نور روز استفاده می‌کنند زیرا انسان‌ها با طبیعت ارتباط دارند. "نور طبیعی را می توان برای نمایش و جان بخشیدن به طراحی استفاده کرد هر ساختمان نور فضایی را در طراحی ساختمان مشخص می کند. نور روز همیشه در نوسان است و اغلب در فضاهای تعاملی ترکیب می شود.
پس وجود نور طبیعی در موزه های تاریخ طبیعی و تنوع زیستی میتواند باعث پیوند بهتر انسان با فضای موزه شود.
مقدار نور روز که در فضای داخلی موزه نفوذ می کند باید مورد توجه جدی قرار گیرد تا بفهمیم نور طبیعی چگونه بر فضا تأثیر می گذارد. عواملی مانند انعکاس، تابش خیره کننده، سازگاری و ترسیم در فضا باید از نزدیک مورد تجزیه و تحلیل قرار گیرند. \lr{IESNA} عواملی را بررسی و ساخته است برای استفاده ی بهتر و همچنین محافظت اشیا از نور روز در فضاهای موزه ای


میتوان با مراجعه به آن قوانین تصمیم بهتری برای طراحی شرایط محیطی موزه گرفت.
در رابطه با نورپردازی در یک موزه تئوری های درست زیادی وجود دارد اما یک قوانین کلی وجود دارد که ما با استفاده از نوع موزه‌ و کارکرد موزه از آن‌ها استفاده میکنیم.


هنگامی که یک نور مصنوعی در موزه نصب می شود، باید تعدادی از عوامل را در نظر گرفت  مشکلات اصلی که باید مورد توجه قرار گیرد عبارتند از:
 • نیاز به محدود کردن روشنایی و کنترل آن به منظور آسیب نرساندن به آثار به نمایش درآمده (حساس به اثرات تابش نور)؛
• امکان ارائه ی یک پروژه نورپردازی خلاقانه با مقداری صرفه جویی در مصرف انرژی.

پروژه ای در رابطه با بهره‌وری انرژی قابل ذکر می باشد، پروژه ساخته شده 
 برای موزه ویلهلم هک در لودویگشافن است، که هدف آن کاهش 70 درصدی هزینه انرژی برای روشنایی‌ است .
عوامل اصلی که استفاده صحیح از روشنایی می تواند منجر به مصرف انرژی کمتر شود عبارتند از:
 • حداکثر استفاده از نور طبیعی و استفاده از منابع مصنوعی تنها به منظور تعادل در استفاده از نور طبیعی؛ 
• استفاده از منابع نوری که حداکثر بازده انرژی را دارند مانند اندازه لامپ ها، رنگ بندی و طول عمر لامپ. 
• قرارگیری صحیح ابزار روشنایی به منظور اطمینان از بیشترین استفاده کافی از نور ساطع شده، اجتناب از تابش خیره کننده و سپس اطمینان از جلوه ی بصری آن.
استفاده از ابزار روشنایی مناسب برای ایجاد دید بهینه و در عین حال اطمینان از ظاهر آن ضروری است. نور ساطع شده از لامپ های \rl{LED} جایگزین مناسبی برای لامپ های رشته ای هالوژنی است. کاهش اتلاف حرارت \rl{LED} ها نیز تأثیر مثبتی بر هزینه ها دارد: گرمایش بار در سیستم های تهویه مطبوع به طور قابل توجهی کمتر است و قدرت سیستم های \rl{HVAC} می تواند باشد به میزان قابل توجهی کاهش یافت. همچنین استفاده از این لامپ ها به طور قابل توجهی خطر محو شدن رنگ را کاهش می دهد و آسیب به مواد حساس. \rl{LED} ها به دلیل عمر بسیار طولانی خود، فواصل نگهداری طولانی تری را فراهم می کنند. طول عمر لامپ های \rl{LED} کار تعمیر و نگهداری گران قیمت را به حداقل می رساند  به خصوص در شرایطی که تغییر لامپ نیاز به تلاش قابل توجهی دارد.


\section{نقش موزه‌‌های تنوع زیستی در آموزش}
موزه تاریخ طبیعی یا موزه تنوع زیستی یک موسسه علمی با مجموعه های تاریخ طبیعی است که شامل سوابق فعلی و تاریخی از حیوانات، گیاهان، قارچ ها، اکوسیستم ها، زمین شناسی، دیرینه شناسی، اقلیم شناسی و غیره است.
نقش اصلی موزه تاریخ طبیعی ارائه نمونه علی کنونی و تاریخی به جامعه علمی برای تحقیقاتشان است، که به منظور بهبود درک ما از جهان طبیعی است.

وظیفه اصلی موزه‌ها آشنایی جامعه و ارتقاء اگاهی هاي عمومی در مورد جانوران.گیاهان و سایر موجودات زنده ،فسیل ها،کانی‌ها و همچنین  
زیستگاه هاي طبیعی است.این مکان‌ها عامل انتقال و توسعه دانش و در دنیاي طبیعی و جایگاه و نقش انسان در ان می باشند. موزه‌ها از طریق  
نمایشگاه ها و برنامه هاي آموزشی که با اهداف خاص تدوین می شوند و به تفسیر و شرح جایگاه و اجزاء طبیعت.ارتباط ان با یکدیگر و با انسان.اهمیت حفظ ان و اهمیت ایجاد ارتباط بین طبیعت و انسان می پردازند.آنها درك انسان را از محیط اطراف خود افزایش می دهند و فرصت هایی را براي کسب دانش و غنی ساختن فرهنگ فراهم می آورند.موزه هاي تاریخ طبیعی می توانند از طریق نمایگاه‌ها و فعالیت هاي اموزشی و اطلاعات نهفته در مجموعه‌ها و در واقع کارشناسان خود دانش لازم را به مردم عادي انتقال دهند موزه هاي امروزي می توانند طرز تفکر مردم را در  
مورد طبیعت تغییر دهند و رفتار و دیدگاه هاي مثبتی را نسبت به محیط طبیعی ترویج نمایند چرا که بهتر زیستن نیاز به دانش زیست محیطی و آموزش دارد و آغاز هر حیاتی مستلزم اهمیت دادن به حیات وحش اب و خاك و منابع طبیعی است. در عین حال بخش هاي علمی و بایگانی موزه هاي تاریخ طبیعی داراي مجموعه هاي وسیعی از جانوران،فسیل‌ها.سنگ‌ها و کانی هاست که می توانند مواد و اطلاعات لازم را جهت تحقیق  
زیست شناسان و دانشجویان علوم طبیعی فراهم آورند.
مجموعه های تاریخ طبیعی مخازن ارزشمندی از اطلاعات ژنومی هستند که می توانند برای بررسی تاریخچه تنوع زیستی و تغییرات محیطی مورد استفاده قرار گیرند. همکاری بین موزه‌ها و محققان در سراسر جهان دانشمندان را قادر می‌سازد تا روابط اکولوژیکی و تکاملی مانند اهلی کردن اسب را با استفاده از نمونه‌های ژنتیکی از مجموعه‌های موزه کشف کنند. روش‌ها و فن‌آوری‌های جدیدی برای حمایت از موزه‌شناسی در حال توسعه هستند.

\section{گونه های در معرض انقراض}
در بحران انقراض یا در سراشیبی انقراض یا در معرض نابودی گونه‌ای، بالاترین درجه ریسک برای جانوران زمین در فهرست سرخ \rl{IUCN} است که توسط اتحادیه بین‌المللی حفاظت از طبیعت به گونه‌ها، با توجه به وضعیت حفاظت آن‌ها، داده می‌شود. در بحران انقراض به این معنا است که جمعیت گونه کاهش یافته یا در طی سه نسل آینده به زیر ۸۰٪ جمعیت کنونی برسد.
از چهار دهه پیش که بشر انقراض بسیاری از گونه‌های زیستی جانوری و گیاهی را در محیط زندگی خود مشاهده می‌کند، تنوع اکوسیستم و حفظ آن برای بقاء حیات در کره زمین اهمیت بیشتری یافته است.
درست است که انقراض بخشی از چرخهٔ طبیعی به‌شمار می‌رود، ولی انسان به انقراض گونه‌ها سرعت بخشیده است و گونه‌های در معرض خطر انقراض را افزایش می‌دهد. به عنوان مثال بنا به گزارش صندوق جهانی حیات وحش در سوئیس گسترش مناطق زیرکشت درختان زودرشد (توسط انسان) به جای جنگل‌های طبیعی مانع شکل گیری چرخه زیستی جانوران می‌شود و به همین دلیل انقراض گونه‌های اکوسیستم را سرعت می‌بخشد.
 یا گونه گیاهی لاله واژگون در ایران به علت چیدن آن‌ها از طبیعت و فروش آن‌ها به عنوان گل زینتی در خطر انقراض است.

پیمان تجارت بین‌المللی گونه‌های جانوری و گیاهان وحشی در معرض خطر

\rl{CITES} (پیمان تجارت بین‌المللی گونه‌های جانوری و گیاهان وحشی در معرض خطر) در سال ۱۹۷۳ میلادی امضا شد. هدف از این معاهده اطمینان از آن است که تجارت حیوانات وحشی و گونه‌های گیاهی ارزشمند و در خطر انقراض، بقای آنها را تهدید نکند. امروز بیش از ۳ هزار گونه جانوری و گیاهی در فهرست حفاظتی \rl{CITES} ثبت شده‌اند و بر تجارت آنها نظارت می‌شود. اجلاس‌های \rl{CITES} دوساله است و در آن تصمیماتی گرفته و قطعنامه‌هایی صادر می‌شود که مقررات اجرایی کنوانسیون هستند و از طرف کشورهای عضو اجرا می‌شوند که برخی از گونه‌های جانوری ایران نیز در فهرست \rl{CITES} قرار دارد.

