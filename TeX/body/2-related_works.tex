% the introduction section...

\chapter{سابقه‌ی تحقیق} \label{chapter:related-works}

\paragraph*{}
\section{تعاریف}

\subsection{موزه تاریخ طبیعی}
اولين موزه تاريخ طبيعي در قرن
هفدهم و در شهر پاريس افتتاح شد.
در ابتدا عموم مردم اجازه بازديد از اينگونه موزه‌ها را نداشتند زيرا يا به
صورت مجموعه هاي خصوصي و يا زير نظر متخصصان علوم طبيعي قرار داشتند. اما به تدريج درهاي آن به
روي مردم باز شد.
وظيفه اصلي موزه هاي تاريخ طبيعي آشنايي جامعه و ارتقاء آگاهي هاي عمومي در مورد جانوران ، گياهان و 
ساير موجودات زنده ، فسيل ها ، سنگ ها ، كاني ها و همچنين زيستگاه هاي طبيعي است . اين مكان ها 
عامل انتقال وتوسعه دانش ودرك دنياي طبيعي وجايگاه ونقش انسان درآن مي باشند . موزه هاي تاريخ طبيعي
از طريق نمايشگاهها وبرنامه هاي آموزشي كه با اهداف خاص تدوين مي شوند به تفسير وشرح جايگاه واجزاء
طبيعت ،ارتباط آنها با يكديگر وبا انسان،اهميت حفظ آن واهميت ايجادارتباط بين طبيعت وانسان مي
پردازند.آنها درك انسان را از محيط اطراف خود افزايش مي دهندوفرصتهايي براي كسب دانش وغني ساختن
فرهنگ فراهم مي آورند.موزه هاي تاريخ طبيعي ميتوانند از طريق نمايشگاهها و فعاليت هاي آموزشي واطلاعات
نهفته درمجموعه ها ودرواقع كارشناسان خود دانش لازم را به مردم عادي انتقال دهند.موزه هاي امروزي مي
توانند طرز تفكر مردم را درمورد طبيعت تغيير دهند ورفتار وديدگاههاي مثبتي را نسبت به محيط طبيعي
ترويج نمايند چرا كه بهتر زيستن نياز به دانش زيست محيطي وآموزش دارد وآغاز هر حياتي مستلزم اهميت
دادن به حيات وحش ،آب وخاك ومنابع طبيعي است . در عين حال بخش هاي علمي و بايگاني موزه هاي تاريخ
طبيعي داراي مجموعه هاي وسيعي از جانوران ، گياهان ، فسيل ها ، سنگ ها و كاني ها است كه مي توانند مواد
و اطلاعات لازم را جهت تحقيق زيست شناسان و دانشجويان علوم طبيعي فراهم آورند.
