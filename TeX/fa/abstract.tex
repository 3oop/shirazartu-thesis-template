% the persian abstract section...

\chapter*{چکیده}

\paragraph*{}
بشر از سپیده‌دم پدیداریش پیوسته مشغول دگرگونی حیات پیرامونش بوده‌است. به حدی که عواقب رفتارش با محیط زیستش اورا مجبور به ترک آن می‌کرد. به مرور با بیشتر شدن علاقه زیست شناسان به مطالعه موجودات زنده موزه های تاریخ طبیعی در قرن 17 و 18 ایجاد شدند. 
نظریه‌های اساسی زیست شناسی با مطالعه همین مجموعه‌ها به سرانجام رسید. فهم ارتباط تکاملی گونه‌ها میسر شد. مقاصد پزوهشی بخشی از کاربرد موزه‌های تاریخ طبیعی هستند. ثبت گونه های زیستی پایش اکوسیستم‌ها را ممکن کرد و به این شکل انسان بهتر توانست تاثیر خود بر محیط اطرافش  را ببیند. تنظیم رابطه انسان و محیط زیست و هدایت مسیر توسعه با کمترین تداخل می‌تواند مهم ترین فایده موزه های تاریخ طبیعی باشد. و در کنار آن شناسایی و مطالعه گونه های کمیاب، به حفظ میراث زیستی هر جغرافیا نیز کمک می‌کند.

موزه تنوع زیستی تهران که اکنون خانه پیروز تنها توله یوز ایرانی محاقظت شده نیز هست از سال 83 فعالیت خود را آغاز کرد. 
در این پژوهش قصد داریم معیار های برای ارزیابی عملکرد موزه معرفی و آن هارا بررسی کنیم.
\subsection*{کلمات کلیدی:}
موزه، موزه تنوع زیستی، موزه تاریخ طبیعی، دیوراما، گونه‌های در معرض انقراض، محیط زیست، تنوع زیستی


اولین موزه تاریخ طبیعی ایران در پردیس علوم دانشگاه تهران در سال 1331 تاسیس شد. امروزه اکثر شهر های بزرگ ایران دارای موزه تاریخ طبیعی هستند.
