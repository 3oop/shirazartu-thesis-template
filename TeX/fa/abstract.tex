% the persian abstract section...

\chapter*{چکیده}

\paragraph*{}
موزه تنوع زیستی در مجموعه پارک پردیسان در کنار ساختمان حفاظت از محیط زیست تاسیس شده و از سال 83 آغاز به فعالیت کرده است.
موزه های تاریخ طبیعی وظیفه آگاه‌سازی و پژوهشی و حفاظت در مورد محیط زیست دارند.


در این پژوهش قصد داریم معیار های برای ارزیابی عملکرد این موزه معرفی و آن هارا بررسی کنیم.
که شامل معیار هایی برای ارزیابی شرایط ظاهری موزه، فعالیت‌های آموزشی و آگاه‌سازی و وظایف موزه در قبال شناخت و حفاظت از محیط زیست می‌شوند.
پس از معرفی و ارزیابی این معیارها پیشنهاداتی برای بهتر شدن عملکرد موزه می‌دهیم.

پرسش‌هایی که در این پژوهش سعی داریم به آن‌ها پاسخ دهیم به این صورت هستند:
\begin{itemize}
    \item آیا موزه از نظر ظاهری و فنی استاندارد است؟
    \begin{itemize}
        \item آیا شیوه نورپردازی در موزه تنوع زیستی با شرایط نگه‌داری نمونه‌ها مطابقت دارد؟
        \item آیا طراحی دیوراماها با زیست بوم نمونه‌ها مطابقت دارد؟
        \item آیا طراحی ساختمان موزه مناسب کارکرد آن است؟
    \end{itemize}
    \item موزه چه فعالیت‌هایی برای آگاه‌سازی در مورد محیط زیست انجام داده‌است؟
    \item موزه چه راهکارهایی برای تاثیرگذاری بهتر در آموزش در نظر گرفته‌است؟
    \item 
\end{itemize}

\subsection*{کلمات کلیدی:}
موزه، موزه تنوع زیستی، موزه تاریخ طبیعی، دیوراما، گونه‌های در معرض انقراض، محیط زیست، تنوع زیستی

