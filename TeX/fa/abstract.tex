% the persian abstract section...

\chapter*{چکیده}

\paragraph*{}
بشر از سپیده‌دم پدیداریش پیوسته مشغول دگرگونی حیات پیرامونش بوده‌است. به حدی که عواقب رفتارش با محیط زیستش اورا مجبور به ترک آن می‌کرد. به مرور با بیشتر شدن علاقه زیست شناسان به مطالعه موجودات زنده موزه های تاریخ طبیعی در قرن 17 و 18 ایجاد شدند. 
نظریه‌های اساسی زیست شناسی با مطالعه همین مجموعه‌ها به سرانجام رسید. فهم ارتباط تکاملی گونه‌ها میسر شد. مقاصد پزوهشی بخشی از کاربرد موزه‌های تاریخ طبیعی هستند. ثبت گونه های زیستی پایش اکوسیستم‌ها را ممکن کرد.

\subsection*{کلمات کلیدی:}
موزه، موزه تنوع زیستی، موزه تاریخ طبیعی، دیوراما، گونه‌های در معرض انقراض، محیط زیست، تنوع زیستی
